\documentclass[a4paper, 10pt]{article}

\usepackage[utf8]{inputenc}
\usepackage[brazilian]{babel}

% The following packages can be found on http:\\www.ctan.org
\usepackage{graphics} % for pdf, bitmapped graphics files
\usepackage{epsfig} % for postscript graphics files
\usepackage{mathptmx} % assumes new font selection scheme installed
\usepackage{times} % assumes new font selection scheme installed
\usepackage{amsmath} % assumes amsmath package installed
\usepackage{amssymb}  % assumes amsmath package installed
\usepackage{xcolor}

\title{\LARGE \bf
O Princípio da Indução e suas Aplicações
}

\author{João Lucas Azevedo Yamin Rodrigues da Cunha \\
		Mariana Alencar do Vale \\
        Vitor Ribeiro Custódio \\
        Pedro Vitor Valença Mizuno}

\begin{document}
\maketitle

\begin{abstract}

Neste trabalho mostraremos diversas utilizações do princípio de indução em Ciência da Computação. São apresentadas duas provas envolvendo indução matemática e estrutural, bem como o desenvolvimento de seu raciocínio.

\end{abstract}

\section{Introdução}

Indução é um princípio importante \textcolor{red}{[...]}

\section{Desenvolvimento das provas}

\subsection{Equivalência entre princípios de indução}

\subsection{Correção do algoritmo de ordenação por inserção}

\section{Prova}

\subsection{Problema 1}

\subsection{Problema 2}

\section{Conclusão}

\end{document}
