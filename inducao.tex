\documentclass[a4paper, 10pt]{article}

\usepackage[utf8]{inputenc}
\usepackage[brazilian]{babel}

% The following packages can be found on http:\\www.ctan.org
\usepackage{graphics} % for pdf, bitmapped graphics files
\usepackage{epsfig} % for postscript graphics files
\usepackage{mathptmx} % assumes new font selection scheme installed
\usepackage{times} % assumes new font selection scheme installed
\usepackage{amsmath} % assumes amsmath package installed
\usepackage{amssymb}  % assumes amsmath package installed
\usepackage{xcolor}
\usepackage{enumerate}

\title{\LARGE \bf
O Princípio da Indução e suas Aplicações
}

\author{João Lucas Azevedo Yamin Rodrigues da Cunha \\
		Mariana Alencar do Vale \\
        Pedro Vitor Valença Mizuno \\
        Vitor Ribeiro Custódio}

\begin{document}
\maketitle

\begin{abstract}

Neste trabalho mostraremos diversas utilizações do princípio de indução em Ciência da Computação. São apresentadas duas provas envolvendo indução matemática e estrutural, bem como o desenvolvimento de seu raciocínio.

\end{abstract}

\section{Introdução}

Indução é um princípio importante \textcolor{red}{[...]}

\section{Desenvolvimento das provas}

\subsection{Equivalência entre princípios de indução}

\subsection{Correção do algoritmo de ordenação por inserção}

No segundo problema proposto pela atividade, devemos provar a correção do \textit{insertion sort}, um algoritmo de ordenação de listas. A definição do algoritmo é explicitada a seguir:

  \begin{equation}
  \label{eq:sort}
      InsertionSort(l) =
      \begin{cases}
          l, \hspace{11.2em}			 	      se \enspace l\,=\,[\,]\\
          Insert(h, InsertionSorte(l'),\enspace se \enspace l\,=\,h\,::\,l'
      \end{cases}
  \end{equation}
  onde Insert é definido como
  \begin{equation}
  \label{eq:insert}
      Insert(x,l) =
      \begin{cases}
          x::[\,],\enspace se\enspace l\,=\,[\,] \\
          x::l,\: \enspace se\enspace l\,=\,h\,::\,l'\enspace e \enspace x\, \leq \, h
      \end{cases}
  \end{equation}

Vamos compreender alguns termos desse algoritmo. Sendo \textit{l} uma lista, esta pode ser definida como:

	\begin{equation*}
		l =
        \begin{cases}
        	[\,]\,,\; \text{se a lista for vazia}\\
            h\: :: \: l',\; \text{onde \textit{h} é o primeiro elemento de \textit{l} e \textit{l'} é o restante da lista} \\
        \end{cases}
	\end{equation*}

Precisamos definir também o que significa uma lista estar ordenada, que é, supostamente, o produto final do \textit{insertion sort}. Seja \textit{l}
uma lista e \textit{E(l)} o conjunto de todos os elementos da lista \textit{l}. Assim,
\begin{align*}
	E\,(\,l\,)\;=\; \{\;a_0,\;a_1,\;a_2,\;...\:,\;a_n\;\}\:,\;n\:=\:|\,l\,|\,-1
\end{align*}
\\
Logo, seja \textit{l} a lista original e \textit{l'} a lista resultante do processo de ordenação de \textit{l}. Dizemos que \textit{l'} está ordenada se

\begin{gather*}
   i\,)\;E(l')=E(l) \\
   ii\,)\;\; a_i \leq a_{i+1}, \forall a_i \: \in E(l') \: , i\in [0,|l|) \\ 
\end{gather*}  
\\
Ou seja, quando:
\begin{enumerate}[i)]
	\item todos os elementos da lista original estiverem presentes na lista ordenada;
    \item quando todo elemento da lista ordenada for menor ou igual a seu sucessor.
\end{enumerate}

Tendo como base estas definições, começamos a pensar o que é necessário para que um algoritmo de ordenação de listas esteja correto. 

\section{Prova}

\subsection{Problema 1}

\subsection{Problema 2}

\section{Conclusão}

\end{document}
