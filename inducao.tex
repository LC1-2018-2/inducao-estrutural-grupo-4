\documentclass[a4paper, 10pt]{article}

\usepackage[utf8]{inputenc}
\usepackage[brazilian]{babel}

% The following packages can be found on http:\\www.ctan.org
\usepackage{graphics} % for pdf, bitmapped graphics files
\usepackage{epsfig} % for postscript graphics files
\usepackage{mathptmx} % assumes new font selection scheme installed
\usepackage{times} % assumes new font selection scheme installed
\usepackage{amsmath} % assumes amsmath package installed
\usepackage{amssymb}  % assumes amsmath package installed
\usepackage{xcolor}
\usepackage{enumerate}
\usepackage{amsfonts}


\title{\LARGE \bf
O Princípio da Indução e suas Aplicações
}

\author{João Lucas Azevedo Yamin Rodrigues da Cunha \\
		Mariana Alencar do Vale \\
        Pedro Vitor Valença Mizuno \\
        Vitor Ribeiro Custódio}

\begin{document}

\maketitle

\begin{abstract}

Neste trabalho mostraremos diversas utilizações do princípio de indução em Ciência da Computação. São apresentadas duas provas envolvendo indução matemática e estrutural, bem como o desenvolvimento de seu raciocínio.\par

\end{abstract}

\section{Introdução}

Indução é um princípio importante \textcolor{red}{[...]}

\section{Desenvolvimento das provas}

\subsection{Equivalência entre princípios de indução}


 No primeiro problema proposto pela atividade, devemos apresentar a equivalência entre os princípios da indução matemática e da indução forte. \par \vspace{5mm}

	Para provarmos a equivalência, precisamos:\vspace{5mm}
\begin{enumerate}[i)]
\item Provar que conseguimos obter os principios da indução forte pela indução fraca;
\item Provar que conseguimos obter os principios da indução fraca pela indução forte;
\end{enumerate}

\vspace{5mm}Primeiro, precisamos definir indução matemática e indução forte
\begin{itemize}
\item Indução fraca\\ 
	O princípio da indução nos permite provar propriedades sobre os números naturais $\mathbb{N}$={1,2,...}, e a partir de alguns exemplos.
    Para provar que uma propriedade \textit{M} sobre os naturais é válida precisamos mostrar duas coisas:
    
	\begin{enumerate} [a)]	
    \item \textbf{Base Indutiva}: mostrar \textit{$n_0$} que satizdaz a propriedade, isto é, provar \textit{M(\textit{$n_0$})};
    \item \textbf{Passo Indutivo}: assumindo que um natural \textit{n} satisfaça a propriedade \textit{M}, precisamos provar que n+1 também satisfa a propriedade, isto é, precisamos construir uma prova de \textit{M(n)} $\rightarrow$ \textit{ M(n+1)};
	\end{enumerate}
    
\item Indução Forte\\
	A indução forte geralmente é utilizada quando não se pode demonstrar com facilidade a veracidade de uma propriedade utilizando a indução fraca.    
    \begin{enumerate}[a)]
    \item \textbf{Base Indutiva}: O princípio da \textbf{Base Indutiva} da indução forte é o mesmo da base fraca
    \item \textbf{Passo Indutivo}: Ao invés de considerarmos verdadeiro \textit{M(n)} para provar \textit{ M(n+1)}, consideramos que $\forall$ k $\le$ n satisfazem a \textit{M(k)} e entao provamos \textit{M(n+1)}, ou seja, \textit{M(1)} $\wedge$ \textit{M(2)} $\wedge$ ... $\wedge$ \textit{M(n)} $\rightarrow$ \textit{ M(n+1)};
    \end{enumerate}
\end{itemize}

Definidas as induções, podemos iniciar a construção da prova; 

A prova da indução consiste em duas etapas, a base indutiva e o passo indutivo.
Como a base indutiva é a mesma, sua prova é trivial.
No caso do passo indutivo, teremos que provar que de indução fraca conseguimos chegar a indução forte e que da indução forte conseguimos chegar a indução fraca.

\begin{enumerate}
\item Neste item, provaremos I.Forte $\Rightarrow$ I.Fraca:

	\\No \textbf{passo indutivo} da Indução Forte, a partir de uma propriedade \textit{M}, assumimos que \textit{M(1)} $\wedge$ \textit{M(2)} $\wedge$ ... $\wedge$ \textit{M(n)} sao verdadeiros para entao obter \textit{M(n+1)}. 
    Disso, podemos  \textit{M(n)}, que faz parte da hipótese da fraca e com isso obtemos que de \textit{M(n)} $\rightarrow$ \textit{M(n+1)}, logo encontramos o passo indutivo da inducao fraca.

\item Neste item, provaremos que I.Fraca $\Rightarrow$ I.Forte 
	\\	No \textbf{passo indutivo} da Indução Fraca, a partir de uma propriedade \textit{M}, assumimos que \textit{M(n)} seja verdadeiro para então chegar em \textit{M(n+1)}
    \\	Sendo \textit{M(n)} verdadeiro, temos, por hipótese, \textit{M(n-1)} $\rightarrow$ \textit{M(n)}. Continuando o processo até que obtemos \textit{M($n_0$)} $\rightarrow$ \textit{M($n_0$ +1)}. Assim temos que \textit{M($n_0$)} $\wedge$ \textit{M($n_0$+1)} $\wedge$ ... $\wedge$ \textit{M(n)} geram o \textit{M(n+1)} verdadeiro.
    Com isso, obtemos a indução forte a partir da fraca.
\end{enumerate}


\subsection{Correção do algoritmo de ordenação por inserção}

No segundo problema proposto pela atividade, devemos provar a correção do \textit{insertion sort}, um algoritmo de ordenação de listas. A definição do algoritmo é explicitada a seguir:

  \begin{equation}
  \label{eq:sort}
      InsertionSort(l) =
      \begin{cases}
          l, \hspace{11.2em}			 	      se \enspace l\,=\,[\,]\\
          Insert(h, InsertionSorte(l'),\enspace se \enspace l\,=\,h\,::\,l'
      \end{cases}
  \end{equation}
  onde Insert é definido como
  \begin{equation}
  \label{eq:insert}
      Insert(x,l) =
      \begin{cases}
          x::[\,],\enspace se\enspace l\,=\,[\,] \\
          x::l,\: \enspace se\enspace l\,=\,h\,::\,l'\enspace e \enspace x\, \leq \, h
      \end{cases}
  \end{equation}

Vamos compreender alguns termos desse algoritmo. Sendo \textit{l} uma lista, esta pode ser definida como:

	\begin{equation*}
		l =
        \begin{cases}
        	[\,]\,,\; \text{se a lista for vazia}\\
            h\: :: \: l',\; \text{onde \textit{h} é o primeiro elemento de \textit{l} e \textit{l'} é o restante da lista} \\
        \end{cases}
	\end{equation*}

Precisamos definir também o que significa uma lista estar ordenada, que é, supostamente, o produto final do \textit{insertion sort}. Seja \textit{l}
uma lista e \textit{E(l)} o conjunto de todos os elementos da lista \textit{l}. Assim,
\begin{align*}
	E\,(\,l\,)\;=\; \{\;a_0,\;a_1,\;a_2,\;...\:,\;a_n\;\}\:,\;n\:=\:|\,l\,|\,-1
\end{align*}
\\
Logo, seja \textit{l} a lista original e \textit{l'} a lista resultante do processo de ordenação de \textit{l}. Dizemos que \textit{l t} foi corretamente ordenada se

\begin{gather*}
   i\,)\;E(l')=E(l) \\
   ii\,)\;\; a_i \leq a_{i+1}, \forall a_i \: \in E(l') \: , i\in [0,|l|) \\ 
\end{gather*}  
\\
Ou seja, quando:
\begin{enumerate}[i)]
	\item todos os elementos da lista original estiverem presentes na lista ordenada;
    \item quando todo elemento da lista ordenada for menor ou igual a seu sucessor.
\end{enumerate}

Tendo como base estas definições, começamos a pensar

\section{Prova}

\subsection{Problema 1}

\subsection{Problema 2}

\section{Conclusão}

\end{document}
