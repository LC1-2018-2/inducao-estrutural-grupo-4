\documentclass[a4paper, 10pt]{article}

\usepackage[utf8]{inputenc}
\usepackage[brazilian]{babel}

% The following packages can be found on http:\\www.ctan.org
\usepackage{graphics} % for pdf, bitmapped graphics files
\usepackage{epsfig} % for postscript graphics files
\usepackage{mathptmx} % assumes new font selection scheme installed
\usepackage{times} % assumes new font selection scheme installed
\usepackage{amsmath} % assumes amsmath package installed
\usepackage{amssymb}  % assumes amsmath package installed
\usepackage{xcolor}
\usepackage{enumerate}
\usepackage{amsfonts}


\title{\LARGE \bf
O Princípio da Indução e suas Aplicações
}

\author{João Lucas Azevedo Yamin Rodrigues da Cunha \\
		Mariana Alencar do Vale \\
        Pedro Vitor Valença Mizuno \\
        Vitor Ribeiro Custódio}

\begin{document}

\maketitle

\begin{abstract}

Neste trabalho mostraremos diversas utilizações do princípio de indução em Ciência da Computação. São apresentadas duas provas envolvendo indução matemática e estrutural, bem como o desenvolvimento de seu raciocínio.\par

\end{abstract}

\section{Introdução}

Neste documento será descrito o desenvolvimento de dois problemas matemáticos utilizando a indução matemática. Indução é um método de prova matemática usado para demonstrar a verdade de um número infinito de proposições dentro do conjunto dos naturais ou, no caso do segundo problema, dentro da estrutura de uma fórmula. A ideia por trás da indução é, através de casos base ou minimais que sabemos/provamos que são verdadeiros, provarmos, de forma recursiva, que a propriedade é válida para todo o conjunto infinito. 

Utilizaremos este método para provar dois problemas. O primeiro será provar a equivalência entre os princípios da indução forte e da indução matemática. No segundo problema, iremos provar a corretude do algoritmo \textit{Insertion Sort}, muito utlizado para ordenação de listas. Além disso, iremos avaliar as principais diferenças de uso e de estrutura na utilização da indução forte, como será demonstrado no problema 1.

\section{Desenvolvimento das provas}

\subsection{Equivalência entre princípios de indução}


 No primeiro problema proposto pela atividade, devemos apresentar a equivalência entre os princípios da indução matemática e da indução forte. \par \vspace{5mm}

	Para provarmos a equivalência, precisamos:\vspace{5mm}
\begin{enumerate}[i)]
\item Provar que conseguimos obter os principios da indução forte pela indução fraca;
\item Provar que conseguimos obter os principios da indução fraca pela indução forte;
\end{enumerate}

\vspace{5mm}Primeiro, precisamos definir indução matemática e indução forte
\begin{itemize}
\item Indução fraca\\ 
	O princípio da indução nos permite provar propriedades sobre os números naturais $\mathbb{N}$={1,2,...}, e a partir de alguns exemplos.
    Para provar que uma propriedade \textit{M} sobre os naturais é válida precisamos mostrar duas coisas:
    
	\begin{enumerate} [a)]	
    \item \textbf{Base Indutiva}: mostrar \textit{$n_0$} que satizdaz a propriedade, isto é, provar \textit{M(\textit{$n_0$})};
    \item \textbf{Passo Indutivo}: assumindo que um natural \textit{n} satisfaça a propriedade \textit{M}, precisamos provar que n+1 também satisfa a propriedade, isto é, precisamos construir uma prova de \textit{M(n)} $\rightarrow$ \textit{ M(n+1)};
	\end{enumerate}
    
\item Indução Forte\\
	A indução forte geralmente é utilizada quando não se pode demonstrar com facilidade a veracidade de uma propriedade utilizando a indução fraca.    
    \begin{enumerate}[a)]
    \item \textbf{Base Indutiva}: O princípio da \textbf{Base Indutiva} da indução forte é o mesmo da base fraca
    \item \textbf{Passo Indutivo}: Ao invés de considerarmos verdadeiro \textit{M(n)} para provar \textit{ M(n+1)}, consideramos que $\forall$ k $\le$ n satisfazem a \textit{M(k)} e entao provamos \textit{M(n+1)}, ou seja, \textit{M(1)} $\wedge$ \textit{M(2)} $\wedge$ ... $\wedge$ \textit{M(n)} $\rightarrow$ \textit{ M(n+1)};
    \end{enumerate}
\end{itemize}

Definidas as induções, podemos iniciar a construção da prova; 

A prova da indução consiste em duas etapas, a base indutiva e o passo indutivo.
Como a base indutiva é a mesma, sua prova é trivial.
No caso do passo indutivo, teremos que provar que de indução fraca conseguimos chegar a indução forte e que da indução forte conseguimos chegar a indução fraca.

\begin{enumerate}
\item Neste item, provaremos I.Forte $\Rightarrow$ I.Fraca:
	\\No \textbf{passo indutivo} da Indução Forte, a partir de uma propriedade \textit{M}, assumimos que \textit{M(1)} $\wedge$ \textit{M(2)} $\wedge$ ... $\wedge$ \textit{M(n)} sao verdadeiros para entao obter \textit{M(n+1)}. 
    Disso, podemos  \textit{M(n)}, que faz parte da hipótese da fraca e com isso obtemos que de \textit{M(n)} $\rightarrow$ \textit{M(n+1)}, logo encontramos o passo indutivo da inducao fraca.

\item Neste item, provaremos que I.Fraca $\Rightarrow$ I.Forte 
	\\	No \textbf{passo indutivo} da Indução Fraca, a partir de uma propriedade \textit{M}, assumimos que \textit{M(n)} seja verdadeiro para então chegar em \textit{M(n+1)}
    \\	Sendo \textit{M(n)} verdadeiro, temos, por hipótese, \textit{M(n-1)} $\rightarrow$ \textit{M(n)}. Continuando o processo até que obtemos \textit{M($n_0$)} $\rightarrow$ \textit{M($n_0$ +1)}. Assim temos que \textit{M($n_0$)} $\wedge$ \textit{M($n_0$+1)} $\wedge$ ... $\wedge$ \textit{M(n)} geram o \textit{M(n+1)} verdadeiro.
    Com isso, obtemos a indução forte a partir da fraca.
\end{enumerate}


\subsection{Correção do algoritmo de ordenação por inserção}
\label{sec:insertion}

No segundo problema proposto pela atividade, devemos provar a correção do \textit{insertion sort}, um algoritmo de ordenação de listas. A definição do algoritmo é explicitada a seguir:

  \begin{equation}
  \label{eq:sort}
      InsertionSort(l) =
      \begin{cases}
          l, \hspace{11.2em}			 	      se \enspace l\,=\,[\,]\\
          Insert(h, InsertionSort(l'),\enspace se \enspace l\,=\,h\,::\,l'
      \end{cases}
  \end{equation}
  onde Insert é definido como
  \begin{equation}
  \label{eq:insert}
      Insert(x,l) =
      \begin{cases}
          x::[\,],\enspace se\enspace l\,=\,[\,] \\
          x::l,\: \enspace se\enspace l\,=\,h\,::\,l'\enspace e \enspace x\, \leq \, h
      \end{cases}
  \end{equation}

Vamos compreender alguns termos desse algoritmo. Sendo \textit{l} uma lista, esta pode ser definida como:

	\begin{equation*}
		l =
        \begin{cases}
        	[\,]\,,\; \text{se a lista for vazia}\\
            h\: :: \: l',\; \text{onde \textit{h} é o primeiro elemento de \textit{l} e \textit{l'} é o restante da lista} \\
        \end{cases}
	\end{equation*}

Precisamos definir também o que significa uma lista estar ordenada, que é, supostamente, o produto final do \textit{insertion sort}. Seja \textit{l}
uma lista e \textit{E(l)} o conjunto de todos os elementos da lista \textit{l}. Assim,
\begin{align*}
	E\,(\,l\,)\;=\; \{\;a_0,\;a_1,\;a_2,\;...\:,\;a_n\;\}\:,\;n\:=\:|\,l\,|\,-1
\end{align*}
\\
Logo, seja \textit{l} a lista original e \textit{l'} a lista resultante do processo de ordenação de \textit{l}. Dizemos que \textit{l t} foi corretamente ordenada se

\begin{gather*}
   i\,)\;E(l')=E(l) \\
   ii\,)\;\; a_i \leq a_{i+1}, \forall a_i \: \in E(l') \: , i\in [0,|l|) \\ 
\end{gather*}
\\
Ou seja, quando:
\begin{enumerate}[i)]
	\item todos os elementos da lista original estiverem presentes na lista ordenada;
    \item quando todo elemento da lista ordenada for menor ou igual a seu sucessor.
\end{enumerate}

Tendo como base estas definições, começamos a desenvolver. Inicialmente, sabemos que utilizaremos indução, pois, além de ser a proposta do problema, o que queremos provar precisa ser verdadeiro para todos os casos
dentro de um conjunto infinito de possibilidades.
	
Para começarmos a indução, iremos separar em dois casos: o caso base e o caso indutivo. O caso base é necessário, pois a partir dele provamos que o algoritmo é válido para pelo menos um caso, depois disso podemos tentar provar para os casos seguintes.\\

O caso base para o algoritmo \textit{insertion sort}, será para quando a lista estiver vazia, ou seja, quando não tiver nenhum elemento. Nesse caso, a lista tem tamanho zero e, segundo a definição do algoritmo, a saída será a própria lista vazia. Assim, satisfaz o item i da definição acima descrita, uma vez que o conjunto dos elementos das listas de entrada e saída são iguais, por ser um conjunto vazio. Esse caso é trivial, uma vez que, por ter nenhum elemento, a lista já está ordenada.

Já no passo indutivo, temos necessariamente que considerar o funcionamento do \textit{Insertion Sort}. Sabemos que o caso em que não temos nenhum elemento já está ordenado, e conseguimos ver também que o caso em que temos apenas um elemento também é simples. Quando uma lista possui um elemento, ela também já está ordenada.

A dificuldade surge quando começamos a trabalhar com dois ou mais elementos. Assim, destinamos nossa atenção às partes do \textit{Insertion Sort}, dentre elas, o próprio \textit{Insert}.
No caso base, tendo uma lista vazia, o \textit{Insert} apenas adiciona o elemento à cabeça da lista. Esse caso já era esperado e não adiciona nenhum resultado à linha de raciocínio. Porém, quando a lista possui apenas um elemento, o \textit{Insert} só adiciona um novo elemento $x$ à cabeça da lista se $x$ for menor ou igual à cabeça já existente. Ou seja, uma vez que uma lista de tamanho 1 já está ordenada, o \textit{Insert} não altera a propriedade "estar-ordenada" da lista ao adicionar um elemento.

Dessa forma, conseguindo provar que, sempre que uma lista já está ordenada, ao adicionar um novo elemento através do \textit{Insert}, a propriedade de ordenação da lista não é alterada, o \textit{Insertion Sort} estará correto.

Assim, podemos separar em mais alguns casos de aplicação do \textit{Insertion Sort}. Quando a lista está vazia, já sabemos que está ordenada. Ao adicionar um elemento à uma lista vazia, também estará ordenada. Agora, se formos adicionar um elemento $x$ ao caso em que já temos uma cabeça $h$, podemos separar em dois casos:

\begin{enumerate}[i)]
	\item Quando $h \geq x$. Nesse cenário, considerando que a lista anterior já estava na ordem crescente (que é o caso quando temos apenas dois elementos), a lista irá se manter ordenada;
    \item Quando $h < x$. Já nesta situação, o processo será reaplicado em $x$, porém tentando adicioná-lo numa lista menor que a original, em que não esteja contido $h$. No caso de dois elementos, essa lista menor será uma lista vazia e permitirá que seja feita a adição de $x$, só que dessa vez no final da lista resultante.
\end{enumerate}

Isso nos leva à dificuldade do \textit{Insert}: quando a adição é de um novo elemento não menor que a cabeça da lista original. 

Pensemos agora do ponto de vista indutivo. Vamos considerar que $P(n)$ é "o \textit{Insert} adiciona corretamente um elemento numa lista de $n$ elementos". Então, $P(n)$ é nossa hipótese de indução e assumimos que está correta. Também sabemos que $P(0)$ e $P(1)$ são verdadeiros. Então vamos pensar a respeito de $P(n+1)$, que é uma lista de tamanho $n$ com a adição de um elemento.

Como $P(n)$ é verdadeiro por hipótese de indução, então uma lista de tamanho $n$ pode ser ordenada corretamente pelo algoritmo \textit{Insertion Sort}. Então iremos adicionar um elemento $x$ a essa lista. Conforme os casos acima comentados, temos duas situações.

Na primeira, onde $h \geq x$, como a lista de tamanho $n$ anterior já estava ordenada, então, segundo o \textit{Insert}, o $x$ será adicionado no início da lista e esta estará ordenada.

Já na segunda situação, o Insert não irá adicionar o $x$ no início da lista, e, recursivamente, vai tentar adicionar à cauda restante, ou seja à lista menos a cabeça $h$. Porém, pela hipótese de indução $P(n)$, em que uma lista de tamanho $n$ consegue ser ordenada, a lista também conseguirá ser ordenada, independente de qual seja a posição em que seja inserido o $x$ e quantas recursões sejam necessárias para que isso aconteça.

Logo, conseguimos cobrir todos os casos de inserção do \textit{Insert}. Portanto, uma vez que o \textit{Insert} está correto, o algoritmo \textit{InsertionSort} também está.

% Para o passo indutivo, consideramos o I(n), de tal forma que, para provarmos a validade do insertion sort, precisamos provar a validade de insertion, assim, dividindo em novos dois casos.
% Similar ao caso da base indutiva do insertion sort, consideramos que n0 será uma lista vazia. Por definição, temos que n0, agora contendo um novo e único membro, estará ordenado, visto que possui apenas um membro. Assim, gerando uma lista ordenada que, por conseguinte, gera uma lista também ordenada no insertion sort.
% No caso da lista não ser vazia, temos dois casos:
% 	O novo elemento do insert é menor que a cabeça da lista.  Nesse caso, validade da ordenação é trivial, visto que não é necessário que a lista seja reordenada, apenas necessitando que seja inserido o novo elemento como nova cabeça da lista.
% 	O novo elemento do insert não é menor que a cabeça da lista, ou seja, está em qualquer posição. Tal elemento será inserido em uma lista de n-1 elementos, de tal forma que, por meio da hipótese de indução, em I(n) um novo elemento estará corretamente inserido, assim, para I(n+1) também é possível dizer o mesmo.
% Assim, como insert pode ser considerado como correto, insertion sort também pode ser considerado correto.


\section{Prova}

\subsection{Problema 1}

Primeiramente, consideramos uma propriedade $P$ a qual podemos aplicar em $n$ números naturais, números naturais estes delimitados por um $n_0$, que é o menor número em que é possível aplicar a propriedade.

A prova da equivalência da base indutiva das induções Forte e Fraca é trivial, visto que seguem o mesmo princípio: demonstrar que $P(n_0)$ é válido, de tal forma que podemos chegar a mesma conclusão usando qualquer um dos métodos de indução.

No caso do passo indutivo, para atingirmos a equivalência, devemos mostrar que a partir da Indução Forte é possível chegar à Fraca e vice-versa.

\textbf{Indução Forte $\implies$ Indução Fraca:}
	Por meio do passo indutivo da Indução Forte, assumimos que para todo $k,\:n_0 \leq k \leq n,\:P(k)$ é verdadeiro, sendo assim, $P(n+1)$ é verdadeiro.
	Dessa hipótese supomos $P(n+1)$ verdadeiro e que $P(k)$ será verdadeiro no domínio delimitado, de tal forma que é possível assumir $P(n)$ verdadeiro, assim, é viável afirmar que $P(n)$ implica em $P(n+1)$, $P(n) \implies P(n+1)$, o qual corresponde ao passo indutivo da Indução Fraca.
	
\textbf{Indução Fraca $\implies$ Indução Forte:}
	Por meio do passo indutivo da Indução Fraca, assumimos que n aplicado em $P(n)$ é verdadeiro, para então provar $P(n+1)$ verdadeiro.
	Seguindo tal hipótese, podemos entender que para $P(n)$ ser considerado verdadeiro, há um $n-1$ tal que $P(n-1)$ é também válido. Tal lógica é possível ser aplicada em $n-1$ e assim por diante, de tal forma que todo $k$, $n_0 \leq k \leq n$, é satisfatório para a propriedade $P$, tal que $P(k)$ é verdadeiro.
	Assim, chegamos à conclusão que 
\begin{equation*}
  P(n_0) \wedge P(n_1) \wedge ... \wedge P(n-2) \wedge P(n-1) \wedge P(n) \implies P(n+1),
\end{equation*}    
    o qual corresponde ao passo indutivo da Indução Forte.

Por fim, concluímos que Indução Forte e Fraca são equivalentes.

\subsection{Problema 2}

Utilizando as definições da seção \ref{sec:insertion}, podemos sintetizar a prova da seguinte forma.
Aplicando a indução estrutural no algoritmo \textit{Insertion Sort}, podemos separar dois casos distintos de listas: o caso base, no qual a lista é vazia, e o caso indutivo, em que a lista é composta por pelo menos um elemento.

No primeiro caso, a solução é trivial. Por não ter nenhum elemento, a lista já está ordenada. O algoritmo, nessa situação, não altera a lista de nenhuma forma. Por isso, o algoritmo está correto para este caso.

No segundo caso, vamos considerar uma proposição $P(n)$: "o algoritmo \textit{Insertion Sort} está correto para uma lista de $n$ elementos". Ou seja, que para uma lista com $n$ elementos, o algoritmo é capaz de ordená-la corretamente. Consideremos $P(n)$ nossa hipótese de indução e, consequentemente, verdadeira.

Como o algoritmo \textit{Insertion Sort} é correto para uma lista de $n$ elementos, queremos provar que também é correto para uma lista de $n+1$ elementos, ou seja, que $P(n+1)$ é verdadeira.

Adicionando, então, um elemento a uma lista de $n$ elementos, temos duas situações possíveis, ambas abordadas na definição de \textit{Insertion Sort} e \textit{Insert}:
\begin{enumerate}
	\item $h \geq x$: Nesse caso, a cabeça da lista original é maior que o elemento $x$ que queremos adicionar. Como a lista original já pode ser ordenada pelo algoritmo, como diz a hipótese de indução, então $x$ não comprometerá a ordenação da lista. Logo, o \textit{Insertion Sort} é correto neste caso;
    \item $h < x$: Nesta situação, a cabeça da lista original é menor que o elemento $x$. Então, pela definição de \textit{Insert}, $x$ não pode ser adicionado ao início da lista. Uma chamada recursiva será feita, e o \textit{Insert} tentará adicionar $x$ na cauda da lista original, ou seja, esta sem sua cabeça. Porém, essa nova lista composta de $x$ e a lista original sem a cabeça possui um total de $n$ elementos. E, por hipótese de indução, $P(n)$, o algoritmo é correto. Então a lista é ordenada.
\end{enumerate}

% Já no segundo caso, quando a lista tem dois elementos, teremos dois casos para verificar. Para verificar ambos os casos, iremos pegar o último elemento da lista. Ou seja, para o primerio caso, numa lista de dois elementos, ao pegar o último elemento e ele for maior que o anterior, o algoritmo já está correto, pois satisfaz as necessidades do mesmo. Agora para o caso que o último elemento não é maior que o anterior. O algortimo irá executar a função de trocar os dois elementos, mantendo verdadeiro o algoritmo e garantindo a corretude.

\section{Conclusão}

Diante dos dois problemas apresentados, foram realizadas aplicações diferentes do mesmo conceito de Indução como solicitado pela especificação da atividade. O desenvolvimento desta permitiu aos integrantes do grupo se familiarizarem mais com a metodologia, o que permitiu, do nosso ponto de vista, uma resolução correta das questões.

\end{document}
